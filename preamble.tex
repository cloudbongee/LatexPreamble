\documentclass{article}

%% common utilities
\usepackage{amsmath,amsthm,amssymb,comment,tikz-cd}
\usepackage{mathabx,epsfig}


\usepackage[T1]{fontenc} %% fonts

\usepackage[subtle]{savetrees} %% saves space
\usepackage{setspace} %% sets spacing
\usepackage{enumitem} %% you can utilize this for adding labels to enums through [label=(\roman*)] 

%%% general formatting
\doublespacing
\fontfamily{georgia}\selectfont

%These lines set all default values to 0
\setlength{\oddsidemargin}{0in}
\setlength{\evensidemargin}{0in}
\setlength{\headsep}{0in}
\setlength{\topmargin}{0in}
\setlength{\headheight}{0in}
\setlength{\parindent}{0pt}

%These lines get the width correct (1 inch is automatic on left...6.5 comes about then from 8.5 minus 1 minus 1)
\setlength{\textwidth}{6.5in}
\setlength{\hoffset}{0in}

%These lines get the height correct (1 inch is automatic on top...9 comes about then from 11 minus 1 minus 1)
\setlength{\textheight}{9in}
\setlength{\voffset}{.5in}

%%%%%%% Generally useful sets
%%%%%%% Some sets 
\newcommand{\Z}{{\mathbb Z}} % integers
\newcommand{\Q}{{\mathbb Q}} % rationals
\newcommand{\R}{{\mathbb R}} % reals
\newcommand{\C}{{\mathbb C}} % complex
\newcommand{\F}{{\mathbb F}} % field

%%%%%%% Stuff I use for real analysis %%%%%%%%%%%%%%%%%%%%%%%%%%%%%%%%

%%%%%%%% the bigcup and bigcap for an index I 
%%%%% I usually utilize J is a subset of I whenever there is a necessity with indexing subsets. 
%%%%% in this case unionn and intersectt are picking elements from the J set.

\newcommand{\union}{\bigcup_{\alpha \in I}}
\newcommand{\unionn}{\bigcup_{\alpha \in J}}
\newcommand{\intersect}{\bigcap_{\alpha \in I}}
\newcommand{\intersectt}{\bigcap_{\alpha \in J}}


%%%%%% Stuff I use for abstract algebra %%%%%%%%%%%%%%%%%%%%%%%%%%%%%%

\DeclareMathOperator{\GL}{GL} % general linear group
\DeclareMathOperator{\Hom}{Hom} % homomorphism
\DeclareMathOperator{\Orb}{Orb} % orbit
\DeclareMathOperator{\Stab}{Stab} % stabilizer
\DeclareMathOperator{\SL}{SL} % Special linear group
\DeclareMathOperator{\Aut}{Aut} % automorphism
\DeclareMathOperator{\Char}{Char} % Ring characteristic
\DeclareMathOperator{\Ord}{Ord} % Order
\newcommand{\Ring}{\mathcal{R}} % Ring (just a fancy R)

\def\acts{\mathrel{\reflectbox{$\righttoleftarrow$}}} %% acts on arrow
 
%%%%%%%%%%%%%%%%%%%%%%%%%%%%%%%%%%%%%%%%%%%%%%%%

\pagenumbering{gobble}

%%%%%% END OF ANY RELEVANT PREAMBLE
